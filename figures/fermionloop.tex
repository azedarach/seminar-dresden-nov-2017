\documentclass{article}

\usepackage[utf8]{inputenc}
\usepackage[T1]{fontenc}

\usepackage{graphicx}

\DeclareGraphicsRule{*}{mps}{*}{}
\DeclareGraphicsExtensions{.pdf}

\usepackage{amsmath}
\usepackage{feynmp}
\usepackage{tabularx}

\pagestyle{empty}

\begin{document}

\newsavebox{\feynmanrules}

\sbox{\feynmanrules}{
  \begin{fmffile}{./fermionloop}
    \fmfset{thin}{.8pt}

    \begin{fmfgraph*}(80,80)
      \fmfkeep{fermionloop}
      \fmfleft{i}
      \fmfright{o}
      \fmf{dashes}{i,v1}
      \fmf{dashes}{v2,o}
      \fmf{plain,left,tension=0.4,label={\scriptsize $f$},label.dist=3pt}{v1,v2}
      \fmf{plain,left,tension=0.4}{v2,v1}
      \fmfv{label={\scriptsize $h$}, label.dist=3pt}{i}
      \fmfv{label={\scriptsize $h$}, label.dist=-0.5pt}{o}
    \end{fmfgraph*}

  \end{fmffile}
}

\begin{center}
\fmfreuse{fermionloop} \\
\vspace{-20pt}
  $m_h^2 \ll \Lambda_{NP}^2$?
\end{center}

\end{document}
