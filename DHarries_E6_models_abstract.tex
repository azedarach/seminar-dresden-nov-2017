\documentclass[12pt,a4paper]{article}

\usepackage[utf8x]{inputenc}
\usepackage[T1]{fontenc}
\usepackage{lmodern}

\usepackage{amsmath,amssymb}

\pagestyle{empty}

\begin{document}
\section{Naturalness, Dark Matter and $E_6$ Inspired SUSY Models}
The unknown nature of dark matter and other unsolved problems point to
the need for Beyond the Standard Model physics. Supersymmetric models are
among the most highly motivated extensions of the Standard Model.  However,
the results of extensive searches for new physics suggest that many
preferred scenarios predicted in the minimal supersymmetric standard
model (MSSM) are not realised.  The fact that these pre-LHC expectations,
which were commonly guided by ideas about naturalness, have gone unmet raises
the question as to whether the MSSM is fine-tuned.  This possibility is
one of the motivations for studying non-minimal supersymmetric models.  In this
talk, we will explore several extensions of the MSSM that can arise from an
underlying $E_6$ gauge symmetry.  We first discuss the general problem of
quantifying fine-tuning, relating traditional measures to a rigorous approach
based on Bayesian statistics.  We then investigate a new source of
fine-tuning in the simplest $E_6$ inspired models, associated with the
presence of a massive $Z^\prime$ boson.  Finally, we study potential dark
matter candidates in one recently proposed model, with a focus on the
implications of current and future direct detection searches.

\end{document}
